% Template:     Template Controles LaTeX
% Documento:    Funciones para insertar elementos
% Versión:      4.3.3 (27/07/2021)
% Codificación: UTF-8
%
% Autor: Pablo Pizarro R.
%        Facultad de Ciencias Físicas y Matemáticas
%        Universidad de Chile
%        pablo@ppizarror.com
%
% Sitio web:    [https://latex.ppizarror.com/controles]
% Licencia MIT: [https://opensource.org/licenses/MIT]

% Insertar párrafo
\newcommand{\newp}{%
	\hbadness=10000 \vspace{\baselineskip} \par%
}

% Insertar párrafo
% 	#1	Párrafo
\newcommand{\newpar}[1]{%
	\hbadness=10000 #1 \newp%
}

% Insertar párrafo sin nueva línea al final
% 	#1	Párrafo
\newcommand{\newparnl}[1]{%
	#1 \par%
}

% Crea un salto de columna en el entorno multicol
\newcommand{\newcolumn}{%
	\checkinsidemulticol\vfill\null\columnbreak%
}

% Salto de página en entorno multicol
\newcommand{\newpagemulticol}{%
	\newcolumn\newcolumn\clearpage%
}

% Redimensiona un ítem
% 	#1	Tamaño del nuevo objeto (En linewidth)
%	#2	Objeto a redimensionar
\newcommand{\itemresize}[2]{%
	\emptyvarerr{\itemresize}{#1}{Tamano del nuevo objeto no definido}%
	\emptyvarerr{\itemresize}{#2}{Objeto a redimensionar no definido}%
	\resizebox{#1\linewidth}{!}{#2}%
}

% Crea una página vacía sin header o footer
\newcommand{\insertemptypage}{%
	\clearpage
	\setcounter{templatePageCounter}{\thepage}
	\pagenumbering{gobble}
	\null
	\thispagestyle{empty}
	\clearpage
	\pagenumbering{arabic}
	\setcounter{page}{\thetemplatePageCounter}
}

% Inserta una página vacía, aunque conserva header, footer y numeración
\newcommand{\insertblankpage}{%
	\clearpage
	\null
	\clearpage
}

% Añade un archivo pdf con el header
%	#1	Parámetros (opcional)
%	#2	Nombre del archivo pdf
\newcommand{\includehfpdf}[2][]{%
	\includepdf[pagecommand={\pagestyle{fancy}},#1]{#2}%
}

% Añade un archivo pdf con el header
%	#1	Parámetros (opcional)
%	#2	Nombre del archivo pdf
\newcommand{\includefullhfpdf}[2][]{%
	\includepdf[pages=-,pagecommand={\pagestyle{fancy}},#1]{#2}%
}

% Inserta un texto entre comillas
%	#1 	Texto
\newcommand{\quotes}[1]{%
	\enquote*{#1}%
}

% Inserta un texto entre comillas dobles
%	#1 	Texto
\newcommand{\doublequotes}[1]{%
	\enquote{#1}%
}

% Inserta una cita con texto elevado
%	#1	Cita
\newcommand{\scite}[1]{%
	\textsuperscript{\cite{#1}}%
}

% Fuerza la indentación
\newcommand{\forceindent}{%
	~ \\ %
	
	\vspace{-2\baselineskip}%
}

% Inserta un texto con el formato de enlace
% 	#1 	Enlace
\newcommand{\hreftext}[1]{%
	\ifthenelse{\equal{\fonturl}{same}}{#1}{\ifthenelse{\equal{\fonturl}{tt}}{\texttt{#1}}{\ifthenelse{\equal{\fonturl}{rm}}{\textrm{#1}}{\ifthenelse{\equal{\fonturl}{sf}}{\textsf{#1}}{}}}}%
}

% Inserta un email con un link cliqueable
%	#1 	Dirección email
\newcommand{\insertemail}[1]{%
	\href{mailto:#1}{\hreftext{#1}}%
}

% Inserta un teléfono celular
%	#1	Teléfono celular
\newcommand{\insertphone}[1]{%
	\href{tel:#1}{\hreftext{#1}}%
}

% Reinicia el número de ecuaciones
\newcommand{\restartequation}{%
	\setcounter{equation}{0}%
}

% Desactiva el margen de las leyendas
\newcommand{\disablecaptionmargin}{%
	\setcaptionmargincm{0}%
}

% Reinicia el margen de las leyendas
\newcommand{\resetcaptionmargin}{%
	\setcaptionmargincm{\captionlrmargin}%
}

% Modifica el color de las tablas
%	#1	Posición inicial del inicio de colores
\newcommand{\settablerowcolors}[1]{%
	\emptyvarerr{\settablerowcolors}{#1}{Posicion de fila no definida}
	
	% Usa colores normales
	\ifthenelse{\equal{\GLOBALtablerowcolorswitch}{false}}{
		\ifthenelse{\equal{\tablerowfirstcolor}{none}}{
			\ifthenelse{\equal{\tablerowsecondcolor}{none}}{
				\rowcolors{#1}{}{}
			}{
				\rowcolors{#1}{\tablerowsecondcolor}{}
			}
		}{
			\ifthenelse{\equal{\tablerowsecondcolor}{none}}{
				\rowcolors{#1}{}{\tablerowfirstcolor}
			}{
				\rowcolors{#1}{\tablerowsecondcolor}{\tablerowfirstcolor}
			}
		}
	% Usa colores alternados
	}{
		\ifthenelse{\equal{\tablerowfirstcolor}{none}}{
			\ifthenelse{\equal{\tablerowsecondcolor}{none}}{
				\rowcolors{#1}{}{}
			}{
				\rowcolors{#1}{}{\tablerowsecondcolor}
			}
		}{
			\ifthenelse{\equal{\tablerowsecondcolor}{none}}{
				\rowcolors{#1}{\tablerowfirstcolor}{}
			}{
				\rowcolors{#1}{\tablerowfirstcolor}{\tablerowsecondcolor}
			}
		}
	}

	% Actualiza el índice previo
	\global\def\GLOBALtablerowcolorindex {#1}
}

% Alterna los colores de las tablas a la última ejecución
\newcommand{\settablerowcolorslast}{
	% Usa colores normales
	\ifthenelse{\equal{\GLOBALtablerowcolorswitch}{false}}{
		\ifthenelse{\equal{\tablerowfirstcolor}{none}}{
			\ifthenelse{\equal{\tablerowsecondcolor}{none}}{
				\rowcolors{\GLOBALtablerowcolorindex}{}{}
			}{
				\rowcolors{\GLOBALtablerowcolorindex}{\tablerowsecondcolor}{}
			}
		}{
			\ifthenelse{\equal{\tablerowsecondcolor}{none}}{
				\rowcolors{\GLOBALtablerowcolorindex}{}{\tablerowfirstcolor}
			}{
				\rowcolors{\GLOBALtablerowcolorindex}{\tablerowsecondcolor}{\tablerowfirstcolor}
			}
		}
	% Usa colores alternados
	}{
		\ifthenelse{\equal{\tablerowfirstcolor}{none}}{
			\ifthenelse{\equal{\tablerowsecondcolor}{none}}{
				\rowcolors{\GLOBALtablerowcolorindex}{}{}
			}{
				\rowcolors{\GLOBALtablerowcolorindex}{}{\tablerowsecondcolor}
			}
		}{
			\ifthenelse{\equal{\tablerowsecondcolor}{none}}{
				\rowcolors{\GLOBALtablerowcolorindex}{\tablerowfirstcolor}{}
			}{
				\rowcolors{\GLOBALtablerowcolorindex}{\tablerowfirstcolor}{\tablerowsecondcolor}
			}
		}
	}
}

% Activa el color de las filas de las tablas
%	#1	Posición inicial del inicio de colores
\newcommand{\enabletablerowcolor}[1][]{
	\ifx\hfuzz#1\hfuzz
		\settablerowcolors{2}
	\else
		\settablerowcolors{#1}
	\fi
}

% Desactiva el color de las filas de las tablas
\newcommand{\disabletablerowcolor}{\rowcolors{2}{}{}}

% Alterna los colores de las filas de las tablas
\newcommand{\switchtablerowcolors}{
	\ifthenelse{\equal{\GLOBALtablerowcolorswitch}{false}}{
		\global\def\GLOBALtablerowcolorswitch {true}
	}{
		\global\def\GLOBALtablerowcolorswitch {false}
	}
	\settablerowcolorslast
}

% Actualiza el padding de las celdas de las tablas
%	#1	Padding horizontal (em)
%	#2	Padding vertical (em)
\newcommand{\settablecellpadding}[2]{
	\emptyvarerr{\settablecellpadding}{#1}{Padding horizontal no definido}
	\emptyvarerr{\settablecellpadding}{#2}{Padding vertical no definido}
	\setlength{\tabcolsep}{#1 em} % Horizontal
	\def\arraystretch {#2} % Vertical
}

% Resetea el padding de las celdas de las tablas
\newcommand{\resettablecellpadding}{
	\settablecellpadding{\tablepaddingh}{\tablepaddingv}
}

% Cambia el tamaño de la página
%	#1	Orientacion de la página, puede ser 0 o 90. Por defecto es cero
%	#2	Ancho de la página (cm)
%	#3	Alto de la página (cm)
\newcommand{\changepagesize}[3][]{
	% \emptyvarerr{\changepagesize}{#1}{Orientacion de la pagina}
	\emptyvarerr{\changepagesize}{#2}{Ancho de la pagina no definida}
	\emptyvarerr{\changepagesize}{#3}{Altura de la pagina no definida}
	\ifthenelse{\equal{\compilertype}{lualatex}}{
		\throwwarning{Funcion no valida en compilador lualatex}
	}{
		\clearpage
		\ifthenelse{\equal{#1}{}}{
			\newgeometry{left=\pagemarginleft cm, top=\pagemargintop cm, right=\pagemarginright cm, bottom=\pagemarginbottom cm, paperwidth=#2 cm, paperheight=#3 cm}
		}{
		\ifthenelse{\equal{#1}{0}}{
			\newgeometry{left=\pagemarginleft cm, top=\pagemargintop cm, right=\pagemarginright cm, bottom=\pagemarginbottom cm, paperwidth=#2 cm, paperheight=#3 cm}
		}{
		\ifthenelse{\equal{#1}{90}}{
			\newgeometry{left=\pagemarginleft cm, top=\pagemargintop cm, right=\pagemarginright cm, bottom=\pagemarginbottom cm, paperwidth=#3 cm, paperheight=#2 cm}
		}{
			\throwbadconfig{Orientacion de pagina no valido}{\changepagesize}{0,90}}}
		}
	}
}

% Ofrece diferentes formatos de pagina
% https://www.prepressure.com/library/paper-size
%	#1	Indica la rotación, puede ser 0 o 90
%	#2	Formato de la pagina
\newcommand{\changepagesizeformat}[2][]{%
	\emptyvarerr{\changepagesizeformat}{#2}{Formato de pagina no definido}
	\ifthenelse{\equal{#2}{4A0}}{
		\changepagesize[#1]{168.2}{237.8}
	}{
	\ifthenelse{\equal{#2}{2A0}}{
		\changepagesize[#1]{118.9}{168.2}
	}{
	\ifthenelse{\equal{#2}{A0}}{
		\changepagesize[#1]{84.1}{118.9}
	}{
	\ifthenelse{\equal{#2}{A1}}{
		\changepagesize[#1]{59.4}{84.1}
	}{
	\ifthenelse{\equal{#2}{A2}}{
		\changepagesize[#1]{42.0}{84.1}
	}{
	\ifthenelse{\equal{#2}{A3}}{
		\changepagesize[#1]{29.7}{42.0}
	}{
	\ifthenelse{\equal{#2}{A4}}{
		\changepagesize[#1]{21.0}{29.7}
	}{
	\ifthenelse{\equal{#2}{A5}}{
		\changepagesize[#1]{14.8}{21.0}
	}{
	\ifthenelse{\equal{#2}{A6}}{
		\changepagesize[#1]{10.5}{14.8}
	}{
	\ifthenelse{\equal{#2}{letter}}{
		\changepagesize[#1]{21.59}{27.94}
	}{
	\ifthenelse{\equal{#2}{legal}}{
		\changepagesize[#1]{21.59}{35.6}
	}{
	\ifthenelse{\equal{#2}{foolscap}}{
		\changepagesize[#1]{20.3}{33.0}
	}{
	\ifthenelse{\equal{#2}{executive}}{
		\changepagesize[#1]{18.41}{26.67}
	}{
	\ifthenelse{\equal{#2}{ledger}}{
		\changepagesize[#1]{27.94}{43.18}
	}{
	\ifthenelse{\equal{#2}{tabloid}}{
		\changepagesize[#1]{43.18}{27.94}
	}{
	\ifthenelse{\equal{#2}{ANSIC}}{
		\changepagesize[#1]{55.9}{43.2}
	}{
	\ifthenelse{\equal{#2}{ANSID}}{
		\changepagesize[#1]{86.4}{55.9}
	}{
	\ifthenelse{\equal{#2}{ANSIE}}{
		\changepagesize[#1]{111.8}{86.4}
	}{
	\ifthenelse{\equal{#2}{B0}}{
		\changepagesize[#1]{100}{141.4}
	}{
	\ifthenelse{\equal{#2}{B1}}{
		\changepagesize[#1]{70.7}{100}
	}{
	\ifthenelse{\equal{#2}{B2}}{
		\changepagesize[#1]{50}{70.7}
	}{
	\ifthenelse{\equal{#2}{B3}}{
		\changepagesize[#1]{35.3}{50}
	}{
	\ifthenelse{\equal{#2}{B4}}{
		\changepagesize[#1]{25}{35.3}
	}{
	\ifthenelse{\equal{#2}{B5}}{
		\changepagesize[#1]{17.6}{25}
	}{
	\ifthenelse{\equal{#2}{B6}}{
		\changepagesize[#1]{12.5}{17.6}
	}{
		\throwbadconfig{Estilo de pagina no valido}{\changepagesizeformat}{4A0,2A0,A0,A1,A2,A3,A4,A5,A6,letter,legal,foolscap,executive,ledger,tabloid,ANSIC,ANSID,ANSIE,B0,B1,B2,B3,B4,B5,B6}}}}}}}}}}}}}}}}}}}}}}}}}
	}
}
