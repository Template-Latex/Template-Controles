% Template:     Control LaTeX
% Documento:    Funciones matemáticas
% Versión:      5.3.0 (07/03/2024)
% Codificación: UTF-8
%
% Autor: Pablo Pizarro R.
%        pablo@ppizarror.com
%
% Manual template: [https://latex.ppizarror.com/controles]
% Licencia MIT:    [https://opensource.org/licenses/MIT]

% Insertar sub-índice, a_b
% 	#1	Elemento inferior (a)
%	#2	Elemento superior (b)
\newcommand{\lpow}[2]{%
	\ensuremath{{#1}_{#2}}
}

% Insertar elevado, a^b
% 	#1	Elemento inferior (a)
%	#2	Elemento superior (b)
\newcommand{\pow}[2]{%
	\ensuremath{{#1}^{#2}}
}

% Inserta inverso función seno, sin^-1
%	#1	Elemento
\newcommand{\aasin}[1][]{%
	\ifx\hfuzz#1\hfuzz%
		\ensuremath{\sin^{-1}#1}
	\else%
		\ensuremath{{\sin}^{-1}}
	\fi%
}

% Inserta inverso función coseno, cos^-1
%	#1	Elemento
\newcommand{\aacos}[1][]{%
	\ifx\hfuzz#1\hfuzz%
		\ensuremath{\cos^{-1}#1}
	\else%
		\ensuremath{\cos^{-1}}
	\fi%
}

% Inserta inverso función tangente, tan^-1
%	#1	Elemento
\newcommand{\aatan}[1][]{%
	\ifx\hfuzz#1\hfuzz%
		\ensuremath{\tan^{-1}#1}
	\else%
		\ensuremath{\tan^{-1}}
	\fi%
}

% Inserta inverso función cosecante, csc^-1
%	#1	Elemento
\newcommand{\aacsc}[1][]{%
	\ifx\hfuzz#1\hfuzz%
		\ensuremath{\csc^{-1}#1}
	\else%
		\ensuremath{\csc^{-1}}
	\fi%
}

% Inserta inverso función secante, sec^-1
%	#1	Elemento
\newcommand{\aasec}[1][]{%
	\ifx\hfuzz#1\hfuzz%
		\ensuremath{\sec^{-1}#1}
	\else%
		\ensuremath{\sec^{-1}}
	\fi%
}

% Inserta inverso función cotangente, cot^-1
%	#1	Elemento
\newcommand{\aacot}[1][]{%
	\ifx\hfuzz#1\hfuzz%
		\ensuremath{\cot^{-1}#1}
	\else%
		\ensuremath{\cot^{-1}}
	\fi%
}

% Fracción de derivadas parciales af/ax
% 	#1	Función a derivar (f)
%	#2	Variable a derivar (x)
\newcommand{\fracpartial}[2]{%
	\ensuremath{\pdv{#1}{#2}}
}

% Fracción de derivadas parciales dobles a^2f/ax^2
% 	#1	Función a derivar (f)
%	#2	Variable a derivar (x)
\newcommand{\fracdpartial}[2]{%
	\ensuremath{\pdv[2]{#1}{#2}}
}

% Fracción de derivadas parciales en n, a^nf/ax^n
% 	#1	Función a derivar (f)
%	#2	Variable a derivar (x)
%	#3	Orden (n)
\newcommand{\fracnpartial}[3]{%
	\ensuremath{\pdv[#3]{#1}{#2}}
}

% Fracción de derivadas df/dx
% 	#1	Función a derivar (f)
%	#2	Variable a derivar (x)
\newcommand{\fracderivat}[2]{%
	\ensuremath{\dv{#1}{#2}}
}

% Fracción de derivadas dobles d^2/dx^2
% 	#1	Función a derivar (f)
%	#2	Variable a derivar (x)
\newcommand{\fracdderivat}[2]{%
	\ensuremath{\dv[2]{#1}{#2}}
}

% Fracción de derivadas en n d^nf/dx^n
% 	#1	Función a derivar (f)
%	#2	Variable a derivar (x)
%	#3	Orden de la derivada (n)
\newcommand{\fracnderivat}[3]{%
	\ensuremath{\dv[#3]{#1}{#2}}
}

% Llave superior de equivalencia
% 	#1	Elemento a igualar
%	#2	Igualdad
\newcommand{\topequal}[2]{%
	\ensuremath{\overbrace{#1}^{\mathclap{#2}}}
}
\newcommand{\topequaltext}[2]{%
	\topequal{#1}{\text{#2}}
}

% Llave inferior de equivalencia
% 	#1	Elemento a igualar
%	#2	Igualdad
\newcommand{\underequal}[2]{%
	\ensuremath{\underbrace{#1}_{\mathclap{#2}}}
}
\newcommand{\underequaltext}[2]{%
	\underequal{#1}{\text{#2}}
}

% Rectángulo superior de equivalencia
% 	#1	Elemento a igualar
%	#2	Igualdad
\newcommand{\topsequal}[2]{%
	\ensuremath{\overbracket{#1}^{\mathclap{#2}}}
}
\newcommand{\topsequaltext}[2]{%
	\topsequal{#1}{\text{#2}}
}

% Rectángulo inferior de equivalencia
% 	#1	Elemento a igualar
%	#2	Igualdad
\newcommand{\undersequal}[2]{%
	\ensuremath{\underbracket{#1}_{\mathclap{#2}}}
}
\newcommand{\undersequaltext}[2]{%
	\undersequal{#1}{\text{#2}}
}

% Función piso
% 	#1	Elemento
\newcommand{\floorexp}[1]{%
	\ensuremath{\left\lfloor{#1}\right\rfloor}
}

% Función techo
% 	#1	Elemento
\newcommand{\ceilexp}[1]{%
	\ensuremath{\left\lceil{#1}\right\rceil}
}

% Función mod
%	#1	Elemento tal que (mod #1)
\newcommand{\Mod}[1]{%
	\ensuremath{\ (\mathrm{mod}\ #1)}
}

% Paréntesis grande
% 	#1	Expresión
\newcommand{\bigp}[1]{%
	\ensuremath{\big(#1\big)}
}

% Paréntesis g+grande
% 	#1	Expresión
\newcommand{\biggp}[1]{%
	\ensuremath{\bigg(#1\bigg)}
}

% Cajón grande
% 	#1	Expresión
\newcommand{\bigc}[1]{%
	\ensuremath{\big[#1\big]}
}

% Cajón g+grande
% 	#1	Expresión
\newcommand{\biggc}[1]{%
	\ensuremath{\bigg[#1\bigg]}
}

% Llave grande
% 	#1	Expresión
\newcommand{\bigb}[1]{%
	\ensuremath{\big\{#1\big\}}
}

% Llave g+grande
% 	#1	Expresión
\newcommand{\biggb}[1]{%
	\ensuremath{\bigg\{#1\bigg\}}
}

% Expresión divergencia
\newcommand{\divexp}{%
	\ensuremath{\rm{div}\ }
}

% Expresión automorfismo
\newcommand{\Autexp}{%
	\ensuremath{\rm{Aut}}
}

% Negrita introducida por word
% 	#1	Expresión
\newcommand{\mathbit}[1]{%
	\bm{#1}
}

% Expresión diff
\newcommand{\Diffexp}{%
	\ensuremath{\rm{Diff}}
}

% Expresión imaginario
\newcommand{\Imexp}{%
	\ensuremath{\rm{Im}}
}

% Expresión imaginario en z
\newcommand{\Imzexp}{%
	\ensuremath{\rm{Im}(z)}
}

% Expresión real
\newcommand{\Reexp}{%
	\ensuremath{\rm{Re}}
}

% Expresión real en z
\newcommand{\Rezexp}{%
	\ensuremath{\rm{Re}(z)}
}

% Barra superior en elemento
%	#1 	Elemento
\newcommand{\overbar}[1]{%
	\mkern 1.5mu\overline{\mkern-1.5mu#1\mkern-1.5mu}\mkern 1.5mu
}

% Función \tilde{} pero que encierra todo el texto
%	#1 	Elemento
\makeatletter
\def\longtilde#1{%
	\mathop{\vbox{\m@th\ialign{##\crcr\noalign{\kern3\p@}%
	\sortoftildefill\crcr\noalign{\kern3\p@\nointerlineskip}%
	$\hfil\displaystyle{#1}\hfil$\crcr}}}\limits%
}
\def\sortoftildefill {%
	$\m@th \setbox\z@\hbox{$\braceld$}%
	\braceld\leaders\vrule \@height\ht\z@ \@depth\z@\hfill\braceru$%
}
\makeatother

% Definición de letras
\newcommand{\A}{\ensuremath{\mathcal{A}}}

\newcommand{\B}{\ensuremath{\mathcal{B}}}

\ifthenelse{\isundefined{\C}}{\newcommand{\C}{C}}{\let\oldC=\C}
\renewcommand{\C}{\ensuremath{\mathbb{C}}}

\newcommand{\D}{\ensuremath{\mathbb{D}}}

\newcommand{\E}{\ensuremath{\mathbb{E}}}

\newcommand{\F}{\ensuremath{\mathcal{F}}}

\ifthenelse{\isundefined{\G}}{\newcommand{\G}{G}}{\let\oldG=\G}
\renewcommand{\G}{\ensuremath{\mathcal{G}}}

\ifthenelse{\isundefined{\H}}{\newcommand{\H}{H}}{\let\oldH=\H}
\renewcommand{\H}{\ensuremath{\mathcal{H}}}

\newcommand{\I}{\ensuremath{\mathbb{I}}}

\newcommand{\J}{\ensuremath{\mathcal{J}}}

\newcommand{\K}{\ensuremath{\mathcal{K}}}

\let\oldL=\L % L con una raya
\renewcommand{\L}{\ensuremath{\mathcal{L}}}

\newcommand{\M}{\ensuremath{\mathcal{M}}}

\newcommand{\N}{\ensuremath{\mathbb{N}}}

% \renewcommand{\O}{\ensuremath{\mathbb{O}}} % O equivale a o/oo

\let\oldP=\P % P negra
\renewcommand{\P}{\ensuremath{\mathbb{P}}}

\newcommand{\Q}{\ensuremath{\mathbb{Q}}}

\newcommand{\R}{\ensuremath{\mathbb{R}}}

\let\oldS=\S % Serpiente
\renewcommand{\S}{\ensuremath{\mathcal{S}}}

\newcommand{\T}{\ensuremath{\mathcal{T}}}

\ifthenelse{\isundefined{\U}}{\newcommand{\U}{U}}{\let\oldU=\U}
\renewcommand{\U}{\ensuremath{\mathcal{U}}}

\newcommand{\V}{\ensuremath{\mathcal{V}}}

\newcommand{\W}{\ensuremath{\mathcal{W}}}

\newcommand{\X}{\ensuremath{\mathcal{X}}}

\newcommand{\Y}{\ensuremath{\mathcal{Y}}}

\newcommand{\Z}{\ensuremath{\mathbb{Z}}}

% Definición de operadores matemáticos de asignación (Typeset assigments)
\ifthenelse{\equal{\fontdocument}{step}}{}{% Ya definidos en STEP
	\newcommand{\asteq}{\ensuremath{\mathrel{{*}{=}}}}
	\newcommand{\eqeq}{\ensuremath{\mathrel{{=}{=}}}}
}
\newcommand{\cdoteq}{\ensuremath{\mathrel{{\cdot}{=}}}}
\newcommand{\diveq}{\ensuremath{\mathrel{{/}{=}}}}
\newcommand{\eqast}{\ensuremath{\mathrel{{=}{*}}}}
\newcommand{\eqcdot}{\ensuremath{\mathrel{{=}{\cdot}}}}
\newcommand{\eqdiv}{\ensuremath{\mathrel{{=}{/}}}}
\newcommand{\eqminus}{\ensuremath{\mathrel{{=}{-}}}}
\newcommand{\eqnot}{\ensuremath{\mathrel{{=}{!}}}}
\newcommand{\eqplus}{\ensuremath{\mathrel{{=}{+}}}}
\newcommand{\eqtimes}{\ensuremath{\mathrel{{=}{\times}}}}
\newcommand{\minuseq}{\ensuremath{\mathrel{{-}{=}}}}
\newcommand{\minusminus}{\ensuremath{\mathrel{{-}{-}}}}
\newcommand{\noteq}{\ensuremath{\mathrel{{!}{=}}}}
\newcommand{\pluseq}{\ensuremath{\mathrel{{+}{=}}}}
\newcommand{\plusplus}{\ensuremath{\mathrel{{+}{+}}}}
\newcommand{\timeseq}{\ensuremath{\mathrel{{\times}{=}}}}

% Definición de teoremas y lemas
\makeatletter
	\renewenvironment{proof}[1][\proofname]{%
		\par\pushQED{\qed}%
		\normalfont\topsep6\p@\@plus6\p@\relax\trivlist%
		\item[\hskip\labelsep\scshape\footnotesize#1\@addpunct{.}]%
		\ignorespaces%
	}{%
		\popQED\endtrivlist\@endpefalse%
	}%
\makeatother
