% Template:     Controles LaTeX
% Documento:    Configuración de página
% Versión:      4.4.5 (10/09/2021)
% Codificación: UTF-8
%
% Autor: Pablo Pizarro R.
%        pablo@ppizarror.com
%
% Manual template: [https://latex.ppizarror.com/controles]
% Licencia MIT:    [https://opensource.org/licenses/MIT]

\newcommand{\templatePagecfg}{
	
	% -------------------------------------------------------------------------
	% Numeración de páginas
	% -----------------------------------------------------------------------------
	\clearpage
	\renewcommand{\thepage}{\arabic{page}}
	\setcounter{page}{1}
	\setcounter{section}{0}
	\setcounter{footnote}{0}
	\ifthenelse{\equal{\showsectioncaptioncode}{none}}{ % Numeración de la sección en los objetos CÓDIGO FUENTE
		\def\sectionobjectnumcode {}
	}{
	\ifthenelse{\equal{\showsectioncaptioncode}{sec}}{
		\def\sectionobjectnumcode {\thesection\sectioncaptiondelimiter}
	}{
	\ifthenelse{\equal{\showsectioncaptioncode}{ssec}}{
		\def\sectionobjectnumcode {\thesubsection\sectioncaptiondelimiter}
	}{
	\ifthenelse{\equal{\showsectioncaptioncode}{sssec}}{
		\def\sectionobjectnumcode {\thesubsubsection\sectioncaptiondelimiter}
	}{
	\ifthenelse{\equal{\showsectioncaptioncode}{ssssec}}{
		\ifthenelse{\equal{\showdotaftersnum}{true}}{
			\def\sectionobjectnumcode {\thesubsubsubsection}
		}{
			\def\sectionobjectnumcode {\thesubsubsubsection\sectioncaptiondelimiter}
		}
	}{
	\ifthenelse{\equal{\showsectioncaptioncode}{chap}}{
		\def\sectionobjectnumcode {\thechapter\sectioncaptiondelimiter}
	}{
		\throwbadconfig{Valor configuracion incorrecto}{\showsectioncaptioncode}{none,chap,sec,ssec,sssec,ssssec}
	}}}}}}
	\ifthenelse{\equal{\showsectioncaptioneqn}{none}}{ % Numeración de la sección en los objetos ECUACIONES
		\def\sectionobjectnumeqn {}
	}{
	\ifthenelse{\equal{\showsectioncaptioneqn}{sec}}{
		\def\sectionobjectnumeqn {\thesection\sectioncaptiondelimiter}
	}{
	\ifthenelse{\equal{\showsectioncaptioneqn}{ssec}}{
		\def\sectionobjectnumeqn {\thesubsection\sectioncaptiondelimiter}
	}{
	\ifthenelse{\equal{\showsectioncaptioneqn}{sssec}}{
		\def\sectionobjectnumeqn {\thesubsubsection\sectioncaptiondelimiter}
	}{
	\ifthenelse{\equal{\showsectioncaptioneqn}{ssssec}}{
		\ifthenelse{\equal{\showdotaftersnum}{true}}{
			\def\sectionobjectnumeqn {\thesubsubsubsection}
		}{
			\def\sectionobjectnumeqn {\thesubsubsubsection\sectioncaptiondelimiter}
		}
	}{
	\ifthenelse{\equal{\showsectioncaptioneqn}{chap}}{
		\def\sectionobjectnumeqn {\thechapter\sectioncaptiondelimiter}
	}{
		\throwbadconfig{Valor configuracion incorrecto}{\showsectioncaptioneqn}{none,chap,sec,ssec,sssec,ssssec}
	}}}}}}
	\ifthenelse{\equal{\showsectioncaptionfig}{none}}{ % Numeración de la sección en los objetos FIGURAS
		\def\sectionobjectnumfig {}
	}{
	\ifthenelse{\equal{\showsectioncaptionfig}{sec}}{
		\def\sectionobjectnumfig {\thesection\sectioncaptiondelimiter}
	}{
	\ifthenelse{\equal{\showsectioncaptionfig}{ssec}}{
		\def\sectionobjectnumfig {\thesubsection\sectioncaptiondelimiter}
	}{
	\ifthenelse{\equal{\showsectioncaptionfig}{sssec}}{
		\def\sectionobjectnumfig {\thesubsubsection\sectioncaptiondelimiter}
	}{
	\ifthenelse{\equal{\showsectioncaptionfig}{ssssec}}{
		\ifthenelse{\equal{\showdotaftersnum}{true}}{
			\def\sectionobjectnumfig {\thesubsubsubsection}
		}{
			\def\sectionobjectnumfig {\thesubsubsubsection\sectioncaptiondelimiter}
		}
	}{
	\ifthenelse{\equal{\showsectioncaptionfig}{chap}}{
		\def\sectionobjectnumfig {\thechapter\sectioncaptiondelimiter}
	}{
		\throwbadconfig{Valor configuracion incorrecto}{\showsectioncaptionfig}{none,chap,sec,ssec,sssec,ssssec}
	}}}}}}
	\ifthenelse{\equal{\showsectioncaptiontab}{none}}{ % Numeración de la sección en los objetos TABLAS
		\def\sectionobjectnumtab {}
	}{
	\ifthenelse{\equal{\showsectioncaptiontab}{sec}}{
		\def\sectionobjectnumtab {\thesection\sectioncaptiondelimiter}
	}{
	\ifthenelse{\equal{\showsectioncaptiontab}{ssec}}{
		\def\sectionobjectnumtab {\thesubsection\sectioncaptiondelimiter}
	}{
	\ifthenelse{\equal{\showsectioncaptiontab}{sssec}}{
		\def\sectionobjectnumtab {\thesubsubsection\sectioncaptiondelimiter}
	}{
	\ifthenelse{\equal{\showsectioncaptiontab}{ssssec}}{
		\ifthenelse{\equal{\showdotaftersnum}{true}}{
			\def\sectionobjectnumtab {\thesubsubsubsection}
		}{
			\def\sectionobjectnumtab {\thesubsubsubsection\sectioncaptiondelimiter}
		}
	}{
	\ifthenelse{\equal{\showsectioncaptiontab}{chap}}{
		\def\sectionobjectnumtab {\thechapter\sectioncaptiondelimiter}
	}{
		\throwbadconfig{Valor configuracion incorrecto}{\showsectioncaptiontab}{none,chap,sec,ssec,sssec,ssssec}
	}}}}}}
	\ifthenelse{\equal{\captionnumcode}{arabic}}{ % Código fuente, INCLUIR SECCIÓN
		\renewcommand{\thelstlisting}{\sectionobjectnumcode\arabic{lstlisting}}
	}{
	\ifthenelse{\equal{\captionnumcode}{alph}}{
		\renewcommand{\thelstlisting}{\sectionobjectnumcode\alph{lstlisting}}
	}{
	\ifthenelse{\equal{\captionnumcode}{Alph}}{
		\renewcommand{\thelstlisting}{\sectionobjectnumcode\Alph{lstlisting}}
	}{
	\ifthenelse{\equal{\captionnumcode}{roman}}{
		\renewcommand{\thelstlisting}{\sectionobjectnumcode\roman{lstlisting}}
	}{
	\ifthenelse{\equal{\captionnumcode}{Roman}}{
		\renewcommand{\thelstlisting}{\sectionobjectnumcode\Roman{lstlisting}}
	}{
		\throwbadconfig{Tipo numero codigo fuente desconocido}{\captionnumcode}{arabic,alph,Alph,roman,Roman}}}}}
	}
	\ifthenelse{\equal{\captionnumequation}{arabic}}{ % Ecuaciones, INCLUIR SECCIÓN
		\renewcommand{\theequation}{\sectionobjectnumeqn\arabic{equation}}
	}{
	\ifthenelse{\equal{\captionnumequation}{alph}}{
		\renewcommand{\theequation}{\sectionobjectnumeqn\alph{equation}}
	}{
	\ifthenelse{\equal{\captionnumequation}{Alph}}{
		\renewcommand{\theequation}{\sectionobjectnumeqn\Alph{equation}}
	}{
	\ifthenelse{\equal{\captionnumequation}{roman}}{
		\renewcommand{\theequation}{\sectionobjectnumeqn\roman{equation}}
	}{
	\ifthenelse{\equal{\captionnumequation}{Roman}}{
		\renewcommand{\theequation}{\sectionobjectnumeqn\Roman{equation}}
	}{
		\throwbadconfig{Tipo numero ecuacion desconocido}{\captionnumequation}{arabic,alph,Alph,roman,Roman}}}}}
	}
	\ifthenelse{\equal{\captionnumfigure}{arabic}}{ % Figuras, INCLUIR SECCIÓN
		\renewcommand{\thefigure}{\sectionobjectnumfig\arabic{figure}}
	}{
	\ifthenelse{\equal{\captionnumfigure}{alph}}{
		\renewcommand{\thefigure}{\sectionobjectnumfig\alph{figure}}
	}{
	\ifthenelse{\equal{\captionnumfigure}{Alph}}{
		\renewcommand{\thefigure}{\sectionobjectnumfig\Alph{figure}}
	}{
	\ifthenelse{\equal{\captionnumfigure}{roman}}{
		\renewcommand{\thefigure}{\sectionobjectnumfig\roman{figure}}
	}{
	\ifthenelse{\equal{\captionnumfigure}{Roman}}{
		\renewcommand{\thefigure}{\sectionobjectnumfig\Roman{figure}}
	}{
		\throwbadconfig{Tipo numero figura desconocido}{\captionnumfigure}{arabic,alph,Alph,roman,Roman}}}}}
	}
	\ifthenelse{\equal{\captionnumsubfigure}{arabic}}{ % Subfiguras, NO USAR SECCIONES YA QUE SON HIJAS DE FIGURA
		\renewcommand{\thesubfigure}{\arabic{subfigure}}
	}{
	\ifthenelse{\equal{\captionnumsubfigure}{alph}}{
		\renewcommand{\thesubfigure}{\alph{subfigure}}
	}{
	\ifthenelse{\equal{\captionnumsubfigure}{Alph}}{
		\renewcommand{\thesubfigure}{\Alph{subfigure}}
	}{
	\ifthenelse{\equal{\captionnumsubfigure}{roman}}{
		\renewcommand{\thesubfigure}{\roman{subfigure}}
	}{
	\ifthenelse{\equal{\captionnumsubfigure}{Roman}}{
		\renewcommand{\thesubfigure}{\Roman{subfigure}}
	}{
		\throwbadconfig{Tipo numero subfigura desconocido}{\captionnumsubfigure}{arabic,alph,Alph,roman,Roman}}}}}
	}
	\ifthenelse{\equal{\captionnumtable}{arabic}}{ % Tablas, INCLUIR SECCIÓN
		\renewcommand{\thetable}{\sectionobjectnumtab\arabic{table}}
	}{
	\ifthenelse{\equal{\captionnumtable}{alph}}{
		\renewcommand{\thetable}{\sectionobjectnumtab\alph{table}}
	}{
	\ifthenelse{\equal{\captionnumtable}{Alph}}{
		\renewcommand{\thetable}{\sectionobjectnumtab\Alph{table}}
	}{
	\ifthenelse{\equal{\captionnumtable}{roman}}{
		\renewcommand{\thetable}{\sectionobjectnumtab\roman{table}}
	}{
	\ifthenelse{\equal{\captionnumtable}{Roman}}{
		\renewcommand{\thetable}{\sectionobjectnumtab\Roman{table}}
	}{
		\throwbadconfig{Tipo numero tabla desconocido}{\captionnumtable}{arabic,alph,Alph,roman,Roman}}}}}
	}
	\ifthenelse{\equal{\captionnumsubtable}{arabic}}{ % Subtablas, NO INCLUIR SECCIÓN YA QUE SON HIJAS DE LAS TABLAS
		\renewcommand{\thesubtable}{\arabic{subtable}}
	}{
	\ifthenelse{\equal{\captionnumsubtable}{alph}}{
		\renewcommand{\thesubtable}{\alph{subtable}}
	}{
	\ifthenelse{\equal{\captionnumsubtable}{Alph}}{
		\renewcommand{\thesubtable}{\Alph{subtable}}
	}{
	\ifthenelse{\equal{\captionnumsubtable}{roman}}{
		\renewcommand{\thesubtable}{\roman{subtable}}
	}{
	\ifthenelse{\equal{\captionnumsubtable}{Roman}}{
		\renewcommand{\thesubtable}{\Roman{subtable}}
	}{
		\throwbadconfig{Tipo numero subtabla desconocido}{\captionnumsubtable}{arabic,alph,Alph,roman,Roman}}}}}
	}

	% -------------------------------------------------------------------------
	% Márgenes de páginas y tablas
	% -----------------------------------------------------------------------------
	\setpagemargincm{\pagemarginleft}{\pagemargintop}{\pagemarginright}{\pagemarginbottom}
	\resettablecellpadding

	% -------------------------------------------------------------------------
	% Se define el punto decimal
	% -------------------------------------------------------------------------
	\ifthenelse{\equal{\pointdecimal}{true}}{
		\decimalpoint}{
	}
	
	% -------------------------------------------------------------------------
	% Definición de nombres de objetos
	% -------------------------------------------------------------------------
	\renewcommand{\appendixname}{\nomltappendixsection} % Nombre del anexo (título)
	\renewcommand{\appendixpagename}{\nameappendixsection} % Nombre del anexo en índice
	\renewcommand{\appendixtocname}{\nameappendixsection} % Nombre del anexo en índice
	\renewcommand{\contentsname}{\nomltcont} % Nombre del índice
	\renewcommand{\figurename}{\nomltwfigure} % Nombre de la leyenda de las fig.
	\renewcommand{\listfigurename}{\nomltfigure} % Nombre del índice de figuras
	\renewcommand{\listtablename}{\nomlttable} % Nombre del índice de tablas
	\renewcommand{\lstlistingname}{\nomltwsrc} % Nombre leyenda del código fuente
	\renewcommand{\lstlistlistingname}{\nomltsrc} % Nombre índice código fuente
	\renewcommand{\refname}{\namereferences} % Nombre de las referencias (bibtex)
	\renewcommand{\bibname}{\namereferences} % Nombre de las referencias (natbib)
	\renewcommand{\tablename}{\nomltwtable} % Nombre de la leyenda de tablas
	
	% -------------------------------------------------------------------------
	% Estilo de títulos
	% -------------------------------------------------------------------------
	\sectionfont{\color{\titlecolor} \fontsizetitle \styletitle \selectfont}
	\subsectionfont{\color{\subtitlecolor} \fontsizesubtitle \stylesubtitle \selectfont}
	\subsubsectionfont{\color{\subsubtitlecolor} \fontsizesubsubtitle \stylesubsubtitle \selectfont}
	\titleformat{\subsubsubsection}{\color{\ssstitlecolor} \normalfont \fontsizessstitle \stylessstitle}{\thesubsubsubsection}{1em}{}
	\titlespacing*{\subsubsubsection}{0pt}{3.25ex plus 1ex minus .2ex}{1.5ex plus .2ex}
	\def\bibfont {\fontsizerefbibl} % Tamaño de fuente de las referencias
	
	% -------------------------------------------------------------------------
	% Estilo citas
	% -------------------------------------------------------------------------
	\ifthenelse{\equal{\stylecitereferences}{apacite}}{
		\renewcommand{\BOthers}[1]{\apacitebothers\hbox{}}
	}{}
	
	% -------------------------------------------------------------------------
	% Se crean los header-footer
	% -----------------------------------------------------------------------------
	\fancyheadoffset{0pt}
	\pagestyle{fancy}
	\newcommand{\COREstyledefinition}{
		\fancyhf{} % Headers y footers
		\fancyhead[L]{}
		\fancyhead[R]{}
		\renewcommand{\headrulewidth}{0pt} % Ancho de la barra del header
		\fancyfoot[L]{\small \rm \textit{\documenttitle}} % Footer izq, título
		\fancyfoot[R]{\small \rm \nouppercase{\thepage}} % Footer der, curso
		\renewcommand{\footrulewidth}{0.5pt} % Ancho de la barra del footer
	}
	\COREstyledefinition
	\fancypagestyle{styleportrait}{ % Estilo portada
		\pagestyle{fancy}
		\fancyhf{}
		\renewcommand{\headrulewidth}{0pt}
		\setpagemargincm{\pagemarginleft}{\pagemargintop}{\pagemarginright}{\pagemarginbottom}
		\fancyfoot[L]{\small \rm \textit{\documenttitle}} % Footer izq, título
		\fancyfoot[R]{\small \rm \nouppercase{\thepage}} % Footer der, curso
		\fancyfoot[L]{\small \rm \textit{\documenttitle}} % Footer izq, título
		\fancyfoot[R]{\small \rm \nouppercase{\thepage}} % Footer der, curso
		\renewcommand{\footrulewidth}{0.5pt} % Ancho de la barra del footer
	}
	\thispagestyle{styleportrait} % Encabezado control (título e integrantes)
	\begin{spacing}{1.025}
	\begin{flushleft} % Crea el header
		\hspace{0cm}
		\vspace{-0cm}
		\begin{tabular}{c}
			\hspace{-0.45cm}~
			\begin{minipage}[t]{1\linewidth}
				\vspace{-4.65em}
				\universityname ~ \\
				\universityfaculty ~ \\
				\universitydepartment ~ \\
				\coursecode\ \hfpdashcharstyle\ \coursename ~ \\
				\begin{flushright}%
					\vspace{-6.7em}\nobreak~\coreinsertkeyimage{\universitydepartmentimagecfg}{\universitydepartmentimage} \vspace{0cm} ~
				\end{flushright}
			\end{minipage}
			\\
		\end{tabular}
		~
	\end{flushleft}
	\end{spacing}
	\vspace*{-1.15cm}
	\noindent \rule{1\linewidth}{0.5pt}
	\begin{center}
		\vspace*{0.35cm}
		\huge {\documenttitle} ~ \\
		\ifdefempty{\teachingstaff}{}{
			\vspace{0.2cm}
			\normalsize{\teachingstaff}
		}
		\ifdefempty{\evaluationindication}{}{
			\vspace{0.5cm}
			\normalsize{\evaluationindication}
		}
	\end{center}
	\fancypagestyle{plain}{
		\fancyheadoffset{0pt}
		\COREstyledefinition
	}

	% -------------------------------------------------------------------------
	% Muestra los números de línea
	% -------------------------------------------------------------------------
	\ifthenelse{\equal{\showlinenumbers}{true}}{
		\linenumbers}{
	}
	
	% -------------------------------------------------------------------------
	% Configura el nombre del abstract
	% -------------------------------------------------------------------------
	\ifthenelse{\isundefined{\abstractname}}{
		\newcommand{\abstractname}{\nameabstract}
		\throwwarning{La variable \noexpand\abstractname no existe, lo que indica que la libreria babel no se ha cargado. Si ha desactivado la configuracion \noexpand\usespanishbabel debe cargar manualmente la libreria babel con algun otro idioma, como por ejemplo \noexpand\usepackage[english]{babel}, o bien define en true la configuracion \noexpand\useenglishbabel}
	}{
		\renewcommand{\abstractname}{\nameabstract}
	}

	% -----------------------------------------------------------------------------
	% Establece el estilo de las sub-sub-sub-secciones
	% -----------------------------------------------------------------------------
	\titleclass{\subsubsubsection}{straight}[\subsection]

}
